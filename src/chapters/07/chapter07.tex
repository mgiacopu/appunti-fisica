\documentclass[class=book, crop=false, oneside, 12pt]{standalone}
\usepackage{standalone}
\usepackage{amsmath}
\usepackage{../../style}
% \graphicspath{{./assets/images/}}

% arara: pdflatex: { synctex: yes, shell: yes }
% arara: latexmk: { clean: partial }
\begin{document}

\chapter{Gas ideali e reali}

\section{Leggi di gas, equazioni di stato dei gas ideali}

Un gas è un fluido con le seguenti caratteristiche:
\begin{enumerate}
    \item non ha forma né volume proprio, occupa pertanto tutto il volume a disposizione, per esempio quello del recipiente che lo contiene; 
    \item è comprimibile facilmente, con conseguenti variazioni notevoli di volume, densità e pressione.
\end{enumerate}

Considerata una certa quantità di gas, le variabili termodinamiche più appropriate per descrivere lo stato termodinamico del gas e le eventuali trasformazioni sono la pressione \(p\), il volume \(V\) e la temperatura \(T\).

Consideriamo il gas racchiuso dentro un contenitore di volume \(V\), con un valore della pressione eguale in tutti i punti, se \(V\) non è molto grande. 
Quando il volume del contenitore cambia, come può avvenire se una parte dello stesso è mobile, si realizza uno scambio di lavoro con l'ambiente esterno; inoltre, a seconda del tipo di pareti del contenitore, diatermiche o adiabatiche, è possibile o viene impedito lo scambio di calore con l'ambiente. 
Il gas può dunque compiere trasformazioni in cui scambia soltanto lavoro o calore con l'ambiente, oppure entrambi; in ogni caso il bilancio energetico è regolato dal primo principio della termodinamica.

\subsection{Legge isoterma di Boyle}

Si abbia un gas in equilibrio tennodinamico ad una certa pressione entro un dato volume e a temperatura \(T\): se si fanno variare i valori della pressione e del volume, mantenendo costante la temperatura, si trova che in tutti i possibili stati di equilibrio isotermi vale la relazione:
\begin{equation}
    pV = costante
\end{equation}
\emph{a temperatura costante la pressione è inversamente proporzionale al volume}.

Una trasformazione isoterma tra due stati di equilibrio di un gas si può realizzare, ad esempio, se il contenitore, a pareti diatermiche, è  mantenuto in contatto termico con una sorgente di calore alla temperatura \(T\) e la parete mobile si muove a seguito di una differenza infinitesima di pressione tra gas e ambiente esterno. 
Si hanno condizioni di equilibrio meccanico e termico e possiamo assumere che durante la trasformazione la temperatura sia costante e la pressione del gas sempre eguale a quella esterna.

In un sistema di coordinate cartesiane ortogonali nel piano, con il volume sull'asse delle ascisse e la pressione sull'asse delle ordinate, il luogo dei punti che rappresentano gli stati di equilibrio di un gas a una data temperatura è costituito da un ramo di iperbole
Per ogni temperatura si ha una diversa iperbole e le curve così ottenute si chiamano le isoterme del gas ideale. 

\subsection{Legge isobara di Volta-Gay Lussac}

Se la pressione di un gas durante una trasformazione resta costante, si parla di trasformazione isobara; si verifica che in condizioni isobare il volume varia linearmente con la temperatura:
\begin{equation}
    V = V_0 (1 + \alpha t)
\end{equation}
la temperatura è espressa in gradi Celsius, \(V_0\) è il volume occupato dal gas per \(t=0\) e \(\alpha\) è una costante che varia poco al variare del tipo di gas, detta \emph{coeffience di dilatazione termica}.

Per provare la validità della legge isobara di Volta-Gay Lussac si può mettere il gas in equilibrio termico con diverse sorgenti di calore, mantenendo sempre l'equilibrio meccanico con l'ambiente (pressione interna eguale alla pressione esterna costante) e ogni volta misurare il volume del contenitore, che ha una parete mobile. 
La trasformazione isobara, nel piano \(( p, V )\) già considerato, è rappresentata da un segmento di retta parallelo all'asse dei volumi.

\subsection{Legge isocora di Volta-Gay Lussac}

Se invece si mantiene costante il volume di un gas la pressione risulta funzione lineare della temperatura:
\begin{equation}
    p = p_0 (1 + \beta t)
\end{equation}
Anche ora la temperatura è espressa in gradi Celsius; \(p_0\) è la pressione del gas per \(t = 0\) e \(\beta\) una costante, praticamente indipendente dal tipo di gas.

Una trasformazione a volume costante si dice isocora; nel piano \(( p, V )\) essa è rappresentata da un egmento di retta parallelo all'asse delle pressioni.

La verifica della legge isocora di Volta-Gay Lussac si può eseguire utilizzando il solito contenitore, mantenendo bloccata la parete mobile e misurando la pressione in diversi stati di equilibrio, con il gas in contatto termico con diverse sorgenti di calore.

Ricordiamo quanto detto all'inizio e cioè che il comportamento dei diversi gas è in accordo con le leggi precedenti quanto più ci si avvicina alle condizioni di gas ideale (bassa pressione e alta temperatura). 
Così facendo si osserva anche che le costanti \(\alpha\) e \(\beta\) assumono lo stesso valore per tutti i gas: 
\begin{equation*}
    \alpha = \beta = \frac{1}{273.15} ^{\circ} C^{-1}
\end{equation*}

Posso quindi riscrivere le leggi 
\begin{equation}
    V = V_0 \alpha (\frac{1}{\alpha} + t) = V_0 \alpha T
\end{equation}
\begin{equation}
    p = p_0 \alpha (\frac{1}{\alpha} + t) = p_0 \alpha T
\end{equation}
dove con \(T = \frac{1}{\alpha} + t = 272.15 + t \) è indicata la temperatura misurata in kelvin.

\subsection{Legge di Avogadro}

La legge di Avogadro stabilisce che \emph{volumi eguali di gas diversi, alla stessa temperatura e pressione, contengono lo stesso numero di molecole}.
Essa si riferisce a gas che abbiano un comportamento ideale e quindi obbediscano alle leggi precedentemente enunciate.

Se si considera una massa \(M\) eguale ad \(A\) grammi di gas, quantità che si chiama mole, il numero di Avogadro vale:
\begin{equation*}
    N_A = 6.0221 \cdot 10^{23} molecole/mole
\end{equation*}

Dalla legge di Avogadro discende la definizione della settima unità fondametale, quella della quantità di materia. 
Si chiama mole una Mole quantità di materia che contiene tante entità elementari quanti sono gli atomi contenuti in \(0.012 kg\) dell'isotopo \(^{12} C\) del carbonio.

Come conseguenza della legge di Avogadro \emph{una mole di qualsiasi gas, a una data temperatura e pression, occupa sempre lo stesso volume}.\newline
Si trova che se la pressione è quella atmosferica (\( p_0 = 101325 Pa\)) e la temperatura è \(T_0 = 273.15 K = 0 ^{circ} C \), tale volume vale \(V_m\)viene indicato col nome di \emph{volume molare}. 
Pertanto \(n\) moli occupano un volume pari a \(n V_m\) e in particolare una chilomole occupa \(22.414 m^3\), nelle dette condizioni di pressione e temperatura. 

\subsection{Equazione di stato del gas ideale}

Se consideriamo \(n\) moli di un gas alla pressione atmosferica \(p_0\) e alla temperatura \(T_0 = 273.15 K\), esse occupano, come abbiamo appena detto, il volume \(V_0 = n V\).

Mantenendo costante il volume e portando la temperatura al valore \(T\)
\begin{equation*}
    p_T = p_0 \alpha T
\end{equation*} 

Moltiplicando per \(V_0\) si ha
\begin{equation*}
    p_T V_0 = p_0 V_0 \alpha T = p_0 V_T
\end{equation*}
\(V_0\) e \(p_T\) sono le coordinate termodinamiche in un particolare stato di equilibrio alla temperatura \(T\), 
come lo sono \(p_0\) e \(V_T\) per un altro stato, sempre alla temperatura \(T\).

Otteniamo dunque:
\begin{equation*}
    pV = p_0 V_0 \alpha T = n p_0 V_m \alpha T
\end{equation*}

Il prodotto \(p_0 V_m \alpha\) è una costante universale, che ha lo stesso valore per tutti i gas, e quindi: 

\subsubsection*{Legge di stato del gas ideale}

\begin{equation}
    p V = n R T
\end{equation}
con
\begin{equation*}
    R = p_0 V_m \alpha = 1.01325 \cdot 10^5 \cdot 0.022414 \cdot \frac{1}{273.15} = 8.314 J / mole K
\end{equation*}

Definiamo, sulla base delle tre leggi elementari e della legge di Avogadro, come \emph{gas ideale un sistema le cui coordinate termodinamiche in uno stato di equilibrio obbediscono alla (7.6), detta equazione di stato di un gas ideale.}

L'equazione di stato contiene le leggi precedenti: infatti basta mantenere costante \(T\), \(p\) o \(V\) in (7.6)  e si ottengono le tre leggi isoterma, isobara o isocora. 
Anche la legge di Avogadro è contenuta in (7.6), se \(R\) è una costante universale.

Ricordando che \(n = N / N_A\), con \(N\) numero di molecole del gas, abbiamo 
\begin{equation}
    p V = \frac{N}{N_A} R T = N k_b T
\end{equation}
la costante universale
\begin{equation}
    k_b = \frac{R}{N_A} = \frac{8.314}{6.0221 \cdot 10^23} = 1.3807 \cdot 10^{-23} \udm{J/K}
\end{equation}
è detta costante di Boltzmann.

L'equazione di stato dei gas ideali esprime un comportamento limite, al quale si avvicinano i gas reali quanto più lontana è la loro temperatura da \(T = 0\) e quanto più bassa è la loro pressione ovvero la loro densità, cioè quanto più sono caldi e rarefatti. 
In queste condizioni le differenze di comportamento dei diversi gas praticamente scompaiono e tutti seguono approssimativamente (7.6).

\section{Trasformazioni di un gas, lavoro}

Consideriamo due stati di equilibrio \(A\) e \(B\) di un sistema formato da \(n\) moli di gas ideale. 
Noti i valori della pressione e del volume, dall'equazione di stato (7.6) si ricavano i valori della temperatura:
\begin{equation*}
    T_A = \frac{p_A V_A}{n R}
\end{equation*}
\begin{equation*}
    T_B = \frac{p_B V_B}{n R}
\end{equation*}

Una trasformazione che porti il gas dallo stato \(A\) allo stato \(B\) può svolgersi attraverso stati di equilibrio te rmodinamico ed è rappresentabile nel piano ( \(p , V\)) da una curva continua.
Se invece la trasformazione ha luogo attraverso stati di non equilibrio si usa una rappresentazione simbolica a tratto per indicare che si ignorano i valori delle coordinate durante il processo.

La trasformazione attraverso stati di non equilibrio può realizzarsi in conseguenza di un processo di espansione o compressione rapida, per cui non sussiste né equilibrio meccanico né termico, o per effetto di una espansione o compressione lenta con una differenza di pressione finita così che, pur potendoci essere equilibrio termico, non c'è equilibrio meccanico, oppure a seguito di contatto termico con differenza finita di temperatura.

Quando un gas si espande o viene compresso avviene uno scambio di lavoro che in termini infinitesimi si può scrivere in generale \( d W = p d V\). 
In una trasformazione finita dallo stato \(A\) allo stato \(B\) si avrebbe
\begin{equation}
    W = \int_A^B p(V) d V
\end{equation}
però bisogna fare attenzione perché questa espressione esplicita del lavoro è valida sostanzialmente in due sole situazioni: 
\begin{enumerate}
    \item \emph{la trasformazione è reversibile} e pertanto si può calcolare l'integrale, dato che la pressione è determinata in ogni stato intermedio \(p = p_{gas} = p_{amb}\);
    \item \emph{è nota la pressione esterna} che, per esempio, è costante, caso tipico di quando il processo avviene sotto la presione atmosferica; in questa situazione, anche se la trasformazione non è reversibile, il lavoro è calcolabile ed è dato da
    \begin{equation*}
        W = p_{amb} (V_B - V_A)
    \end{equation*}
    In tutti gli altri casi in cui la pressione non è nota non si può applicare la (7.9).
\end{enumerate}

Ad ogni modo, se la trasformazione è isocora (\(V =\) costante, \(\Delta V = 0\)), il lavoro è sempre nullo; se il gas si espande il volume finale \(V_B\) è maggiore del volume iniziale e il gas compie un lavoro sull'ambiente che secondo la convenzione adottata è positivo; 
se il gas viene compresso, \(V_B < V_A\) e il gas subisce un lavoro (negativo), compiuto dall'ambiente.

Il lavoro, se si può utilizzare (7.9), ha un semplice significato geometrico nel piano (\(p,V\)). 
Nel caso di una trasformazione che passa attraverso stati di equilibrio ed è quindi rappresentabile con una curva continua, la curva \(p = p (V)\) nel piano ( \(p, V\)), il lavoro, in accordo con il significato geometrico dell'operazione di integrazione, è pari all'area compresa tra la curva e l'asse dei volumi. 

\section{Energia interna del gas ideale}

%TODO da revisionare
La dipendenza dell'energia interna di un gas ideale dalle coordinate termodinamiche è stata ricavata analizzando il risultato dell'\emph{esperienza sull'espansione libera, eseguita da Joule}. 

In un contenitore con pareti rigide e diatermiche, diviso in due parti eguali separate da un rubinetto, si trova un gas nella parte sinistra, mentre nella parte destra è stata realizzata una condizione di vuoto. 
Il contenitore è immerso in un calorimetro e la temperatura di equilibrio è \(T\). 
Si apre il rubinetto e si lasci a espandere il gas in tutto il volume a disposizione. 
L'espansione è chiamata libera perché non ci sono forze esterne che agiscono sul gas.
Sperimentalmente si osserva che, comunque si operi, aprendo lentamente o rapidamente il rubinetto, con gas inizialmente ad alta o bassa pressione, la temperatura del liquido calorimetrico alla fine del processo è sempre pari a \(T\), temperatura iniziale di equilibrio.

Il gas quindi non scambia calore con il calorimetro, \(Q = 0\). 
Esso inoltre non scambia lavoro con l'esterno (le pareti del contenitore sono rigide) e pertanto \(W = 0\). 
Dal primo principio segue \(\Delta V = Q - W = 0\): \emph{nell'espansione libera l'energia interna di un gas ideale non varia}. 
Possiamo allora giungere alla seguente conclusione: poiché nel processo la temperatura del gas non cambia, mentre variano pressione e volume (in accordo con (9.1) perché la trasformazione è isoterma, cioè \(p_{in} V_{in} = p_{fin} V_{fin}\)), l'energia interna deve essere funzione soltanto della temperatura. 

Per determinare l'espressione esplicita della funzione \(U(T)\) consideriamo due generici stati di equilibrio \(A\) e \(B\): \(\Delta U = U_B - U_A\) deve essere la stessa qualsiasi trasformazione si scelga, essendo \(U\) una funzione di stato.
Se scegliamo in particolare una frasformazione \(A C\) isocora e una isoterma, si ha:
\begin{equation*}
    \Delta U = U_B - U_A = U_B - U_C + U_C - U_A = U_C - U_A
\end{equation*}
in quanto \(U_B = U_C\) essendo gli stati \(B\) e \(C\) alla stessa temperatura ed \(U\) funzione sole della temperatura.

Applichiamo ora il primo principio alla trasformazione isocora: dato che \(W = 0, \Delta U = Q\), dove \(Q\) è il calore scambiato in condizioni isocore. 
Pertanto 
\begin{equation}
    \Delta U = U_B - U_A = n c_v (T_B - T_A) = n c_v \Delta T 
\end{equation}
\begin{equation}
    \Delta U = U_B - U_A = n \int_{T_A}^{T_B} c_v d T
\end{equation}
a seconda che il calore specifico a volume costante sia indipendente dalla temperatura oppure no. 
Per trasformazioni infinitesime
\begin{equation}
    d U = n c_v d T
\end{equation}
da cui si ricava
\begin{equation}
    c_v = \frac{1}{n} \frac{d U}{d T}
\end{equation}

\section{Studio di alcune trasformazioni}

\subsection{Trasformazioni adiabatiche}

Il gas è racchiuso dentro un contenitore con pareti adiabatiche e quindi può scambiare solo lavoro, per esempio in conseguenza del fatto che una parete è mobile. 
Dal primo principio si ha
\begin{equation*}
    W_{AB} = - \Delta U = - n c_v (T_B -T_A)
\end{equation*}
se la trasformazione avviene tra i due stati di equilibrio \(A\) iniziale e \(B\) finale. 

Se si ha un'espansione adiabatica il lavoro \(W_{AB}\) è positivo e quindi \(\Delta U\) è negativa e \(T_B\) è minore di \(T_A\): il gas si raffredda; se invece si ha una compressione adiabatica, \(W_{AB}<0, \Delta U > 0 , T_B > T_A\), il gas si riscalda. 
Queste variazioni di temperatura sono comunemente sperimentate nelle variazioni rapide di volume di un gas.

Se invece la trasformazione è \emph{adiabatica reversibile}, l'espressione infinitesima del primo principio diviene
\begin{equation*}
    d U + d W = n c_v d T + p d V = 0
\end{equation*}
in quanto possiamo esprimere il lavoro in funzione delle coordinate termodinamiche, appunto perché la trasformazione è reversibile.

Per questa stessa ragione si può utilizzare l'equazione di stato in qualsiasi stato intermedio per esprimere la pressione come \(p = n R T / V\) e si ottiene
\begin{equation*}
    n c_V d T + \frac{n R T}{V} d V = 0
\end{equation*}

Se pariamo le variabili e utilizziamo la relazione di Mayer
\begin{equation*}
    \frac{c_p -c_V}{c_V} \frac{d V}{V} = - \frac{d T}{T} \implies (\gamma - 1) \frac{d V}{V} = - \frac{d T}{T}
\end{equation*}
Questa equazione differenzia le rappresenta la condizione a cui obbediscono le coordinate degli stati di un gas ideale collegati da una trasformazione adiabatica reversibile.

\begin{equation*}
    (\gamma - 1) \ln \frac{V_{B}}{V_A} = \ln \frac{T_A}{T_B} \implies \ln \left(\frac{V_B}{V_A}\right)^{\gamma - 1 } = \ln \frac{T_A}{T_B}
\end{equation*}
L'eguaglianza tra i logaritmi comporta l'eguaglianza tra gli argomenti, per cui
\begin{equation*}
    T_A V_A^{\gamma - 1} = T_B V_B^{\gamma - 1}
\end{equation*}
espressione che dà la relazione tra le coordinate termodinamiche del gas durante una trasformazione adiabatica reversibile. 

Tramite l'equazione di stato si può trasformare la relazione tra \(T\) e \(V\) in una tra \(p\) e \(V\) o tra \(p\) e \(T\) e in conclusione si hanno tre espressioni equivalenti:
\begin{equation}
    T V^{\gamma - 1 } = costante \ , \ pV^{\gamma} = costante \ , \ T p^{\frac{1 - \gamma}{\gamma}}
\end{equation}

Le ( 7.14) sono un esempio di quando si possono esprimere i termini di ( 10.5) in funzione delle coordinate termodinamiche, si ottiene una relazione tra queste che rappresenta l'equazione della trasformazione.
Esse si chiamano infatti le equazioni di una trasformazione adiabatica reversibile di un gas ideale.

In particolare utilizziamo l'equazione \(p V^{\gamma} =  costante\) per rappresentare a trasformazione nel piano di Clapeyron. 
Rispetto alla curva isoterma \(p V = costante\), passante per esempio per il punto rappresentativo dello stato \(A\), la curva adiabatica ha un andamento simile però con pendenza maggiore perché \(y\) è sempre maggiore di 1:  si conferma che \(T_B < T_A\).

Una trasformazione adiabatica reversibile costituisce un caso limite, in quanto per essere reversibile dovrebbe svolgersi molto lentamente, ma ciò introduce difficoltà per mantenere l'adiabaticità. 
Le trasformazioni reali sono irreversibili, in particolare possiamo considerare adiabatica irreversibile una trasformazione che comporta una variazione rapida di volume, così che non ci sia tempo per scambi di calore. 
L'espansione libera di Joule è un altro caso di trasformazione adiabatica irreversibile.

\subsection{Trasformazione isoterma}

Nel caso di una trasformazione isoterma si considera il gas racchiuso in un recipiente che è in contatto termico con una sorgente di calore alla temperatura \(T\). 
Durante la trasformazione la temperatura del gas resta costante al valore \(T\) e abbiamo
\begin{equation*}
    \Delta U = 0 \ , \ Q = W \ , \ p_A V_A = p_B V_B
\end{equation*}

Se la trasformazione è un espansione isoterma \(W_{AB}>0\) e quindi \(Q_{AB}>0\): il gas compie lavoro e assorbe calore. 
Se invece la trasformazione è una compressione isoterma \(W_{AB} < 0 \) e \(Q_{AB} < 0\): il gas subisce lavoro e cede calore.

Qualora la trasformazione sia isoterma reversibile dalla legge dei gas ideali e dal lavoro in una trasformazione finita si ha
\begin{equation}
    W_{AB} = \int_A^B p d V = \int_A^B \frac{n R T}{V} d V = n R T \int_A^B \frac{d V}{V} = n R T \ln \frac{V_{B}}{V_{A}} 
\end{equation}
e questa è anche l'espressione esplicita del calore scambiato.

Si noti che è sempre \(Q \neq 0\): una trasformazione isoterma reversibile comporta sempre uno scambio di calore, a meno che non sia \(T = 0\), condizione che non è mai raggiungibile. 
Una particolare trasformazione isoterma irreversibile è l'espansione libera di Joule. 
Tale trasformazione è insieme adiabatica e isoterma: ciò è possibile solo perché la trasformazione è irreversibile, per una reversibile i due fatti sono ben distinti ed è impossibile che una trasformazione isoterma sia anche adiabatica.

\subsection{Trasformazioni isocore}

Il gas è contenuto in un recipiente diatermico di volume fisso: \(V = costante\) e \(W = 0\); il gas può scambiare solo calore e questo è eguale, per il primo principio, alla variazione dell'energia interna:
\begin{equation*}
    Q = \Delta U = m c_v (T_B - T_A) \ \  c_v = costante
\end{equation*}

Essendo il volume costante dell'equazione di stato:
\begin{equation*}
    \frac{p_A}{T_A} = \frac{p_B}{T_B} \implies \frac{p_A}{p_B} = \frac{T_A}{T_B}
\end{equation*}
Se si cede calore al gas, la sua pressione e la sua temperatura aumentano, mentre se si assorbe calore dal gas pressione e temperatura diminuiscono.

\subsection{Trasformazioni isobare}

Il gas è contenuto ora in un recipiente diatermico con una parete mobile su cui agisce una pressione esterna costante \(p\).
Dall'equazione di stato o da ( 11.4) abbiamo che in una trasformazione isobara
\begin{equation*}
    \frac{v_A}{T_A} = \frac{v_B}{T_B} \implies \frac{v_A}{v_B} = \frac{T_A}{T_B}
\end{equation*}

Il gas può scambiare sia calore che lavoro, dati da 
\begin{equation*}
    Q = n c_p (T_B - T_A)
\end{equation*}
\begin{equation*}
    W = p (V_B - V_A) = p (\frac{n R T_B}{p} - \frac{n R T_A}{p}) = n R (T_B - T_A) 
\end{equation*}
e deve essere sempre \(Q - W = \Delta U = n c_v (T_B - T_A)\).

Se si cede calore al gas, il suo volume e la sua temperatura aumentano e il gas compie lavoro; se si assorbe calore dal gas, volume e temperatura diminuiscono, il gas subisce lavoro.

Una trasformazione isobara si compie mettendo il gas, a temperatura \(T_A\), in contatto termico con una sorgente di calore a temperatura \(T_B\); non essendoci equilibrio termico la trasformazione è irreversibile. 
Invece, per avere una trasformazione reversibile bisogna disporre di una serie infinita di sorgenti, come descritto per le trasformazioni isocore. 

\subsection{Trasformazioni generiche}

Per una trasformazione diversa da quelle precedenti possiamo utilizzare il primo principio nella forma:
\begin{equation}
    d Q = d U + d W = n c_v d T + d W
\end{equation}
È necessario poi esaminare attentamente le condizioni termodinamiche che regolano la trasformazione, per stabilire se essa sia reversibile o irreversibile.

Se è reversibile possiamo utilizzare l'equazione di stato \(p V= n R T\) per il lavoro l'espressione \(d W = p d V \). 
Si tenga presente che per il lavoro conviene verificare se può essere calcolato direttamente per via geometrica, cioè tramite l'area sotto la curva che rappresenta la trasformazione nel piano (\(p , V\)).

\end{document}