\documentclass[class=book, crop=false, oneside, 12pt]{standalone}
\usepackage{standalone}
\usepackage{amsmath}
\usepackage{../../style}
% \graphicspath{{./assets/images/}}

% arara: pdflatex: { synctex: yes, shell: yes }
% arara: latexmk: { clean: partial }
\begin{document}

\chapter{Dinamica del punto}

\section{Legge d'inerzia}
L'interazione del punto con l'ambiente circostante, espressa dal concetto di forza, determina la variazione dello stato di moto.

Galileo, da prove sperimentali, formulò il principio d'inerzia: \emph{un corpo non soggetto a forze rimane nel suo stato di moto}. 
Questo venne riformulato da Newton come prima legge della dinamica: \emph{un corpo permane nel suo stato di quiete o di moto rettilineo uniforme a meno che non intervenga una forza esterna a modificare tale stato}. 
Quindi un corpo può essere in stato di quiete \(v = 0\) oppure in moto (Moto Rettilieo Uniforme) con \(v = costante\).

Dal principio di inerzia si deduce che l'interazione, ossia l'azione di una forza, porta a una variazione della velocità in modulo o in direzione o in entrambi.
La forza dunque è la grandezza che esprime e misura l'interazione tra sistemi fisici. 

Alla forza è associata la nozione di intensità e di direzionalità.
L'effetto di una forza cambia con la direzione. Se più forze agiscono sullo stesso corpo ma il corpo non va in uno stato di moto (ovvero la risultante delle forze è nulla) si dice ce si trova in uno stato di equilibrio.

\section{Legge di Newton }
La formulazione quantitativa del legame tra la forza e lo stato di moto è data dalla legge di Newton:
\begin{equation}
    \overrightarrow{F} = m \overrightarrow{a}
\end{equation}
L'interazione del punto con l'ambiente circostante, espressa tramite la forza \(F\), determina l'accelerazione del punto ovvero la variazione della sua velocità nel tempo; \(m\) rappresenta la massa inerziale del punto (Seconda legge della dinamica).

Il termine massa inerziale è legato al fatto che la massa esprime l'inerzia del punto, cioè la sua resistenza a variare il proprio stato di moto, ossia a modificare la velocità (in modulo, direzione e verso).
Nello studio del corpo, che è sempre rappresentato come punto materiale, possiamo trascurarare le sue dimensioni ma non la sua massa che diventa fondamentale.

La legge può essere espansa come:
\begin{equation}
    \overrightarrow{F} = m \frac{\overrightarrow{v}}{dt} = m \frac{d^2 \overrightarrow{r}}{dt^2}
\end{equation}
da questa possiamo ricavare tutte le proprietà al moto di un punto materiale. Viceversa è possibile dall'accelerazione determinare la forza agente e la massa.

Le leggi di Newton sono valide solo se il moto è studiato in una particolare classe di sistemi di riferimento, i cosiddetti sistemi di riferimento inerziali; altrimenti compaiono nelle formule termini correttivi.
In breve un sistema di riferimento inerziale è un sistema che si muove in moto rettilineo uniforme rispetto a un altro sistema, ossia con velocità costante rispetto ad esso.

\section{Quantità di moto}
Si definisce \emph{quantità di moto} di un punto materiale il vettore:
\begin{equation}
    \overrightarrow{p} = m \overrightarrow{v}
\end{equation}
e ci dice quanta massa è in movimento e di quanto si muove.
Se la massa è costante, l'equazione della seconda legge di Newton si può riscrivere come:
\begin{equation}
    \overrightarrow{F} = \frac{d \overrightarrow{p}}{dt}
\end{equation}

Questa relazione è la forma più generale della legge di Newton, utilizzabile anche se la massa non è costante. 
La massa infatti può cambiare in due casi:
\begin{itemize}
    \item modifica della massa di un sistema macroscopico( approssimabile a punto materiale) come avviene per esempio in un veicolo a motore che brucia carburante o in una scala mobile; 
    \item dipendenza della massa dalla velocità: secondo la teoria della relatività ristretta la massa varia al variare della sua velocità (la variazione di massa è rilevante solo a velocità prossime a \(c\))
\end{itemize}

È utile, quando si parla di quantità di moto, fare riferimento anche alla sua relazione rispetto alla forza.
Dall' equazione precedente possiamo ottenere \(\overrightarrow{F} d t = d \overrightarrow{p}\): vediamo che l'azione di una forza durante un tempo \(d t\) (nel quale la forza agisce) provoca una variazione infinitesima della quantità di moto del punto. 
In termini finiti si ha:
\begin{equation}
    \overrightarrow{I} = \int_0^t \overrightarrow{F} dt = \int_{p_0}^p d \overrightarrow{p} = \overrightarrow{p}-\overrightarrow{p_0} = \Delta \overrightarrow{p}
\end{equation} che definisce il "Teorema dell'impulso".

Il termine vettoriale \(I\), integrale della forza nel tempo, è chiamato impulso della forza. 
La formula precedente esprime che l'impulso di una forza applicata ad un punto materiale provoca la variazione della sua quantità di moto: con \(m\) costante si ha ovviamente:
\begin{equation}
    \overrightarrow{I} = m (\overrightarrow{v} - \overrightarrow{v_0}) = m \Delta v
\end{equation}

La variazione della quantità di moto è tanto maggiore quanto più elevato è il valore dell'impulso ovvero, per una determinata forza costante, quanto maggiore è il tempo in cui agisce la forza.
Il teorema dell'impulso è utilizzabile per calcolare effettivamente \(\Delta \overrightarrow{p}\) solo se si conosce la funzione \(F (t)\), in questo caso basta calcolare l'integrale.
Se \(F = costante\), si ottiene subito
\begin{equation}
    \overrightarrow{F} t = m (\overrightarrow{v} - \overrightarrow{v_0})
\end{equation}
che è come misurare l'area di un rettangolo.
Invece se misuriamo \(\Delta \overrightarrow{p}\) con il teorema della media intergrale possiamo calcolare \(\overrightarrow{F_m}\) della forza agente nell'intervallo di tempo \(t\).

Quando \(\overrightarrow{F}\) è nulla, \(\Delta \overrightarrow{p} = 0\) e pertanto \(\overrightarrow{p} = costante\) : in assenza di forza applicata la quantità di moto di un punto materiale rimane costante o, la quantità di moto si conserva (principio di inerzia). 

\section{Risultante delle forze}
Su un punto materia le possono agire contemporaneamente più forze: si constata che il moto del punto ha luogo come se agisse una sola forza, 
la risultante vettoria le delle forze applicate al punto
\begin{equation}
    \overrightarrow{R} = \overrightarrow{F}_1+\overrightarrow{F}_2+ ... + \overrightarrow{F}_n = \Sigma_i \overrightarrow{F}_i
\end{equation}
è questa una conferma della natura vettoriale dell'equazione di Newton.

Nello studio del moto otteniamo informazioni solo sulla risultante delle forze agenti sul punto stesso \(\overrightarrow{R}\) e non sulle singole forze che concorrono a formare la risultante.
Se \(R = 0\) ed il punto ha velocità nulla, esso rimane in quiete: sono realizzate le condizioni di equilibrio statico del punto. 
Devono quindi essere nulle le componenti stesse della risultante, con riferimento ad un sistema di assi cartesiani.

\subsection{Reazioni vincolari}

Se un corpo, soggetto all'azione di una forza o della risultante non nulla di un'insieme di forze, rimane fermo, dobbiamo dedurre che l'azione della forza provoca una reazione dell'ambiente circostante (reazione vincolare) che si esprime tramite una forza, eguale e contraria alla forza o alla risultante delle forze agenti,
applicata al corpo stesso in modo tale che esso rimanga in quiete.

Si introduce quindi la terza legge di Newton, detta anche legge dell'azione e reazione: 
\emph{se un corpo A esercita una forza sul corpo B, allora il corpo B esercita una forza uguale e contraria}.

Nel caso di un corpo appoggiato su di un tavolo, il corpo è soggetto all'azione di attrazione della terra, perpendicolarmente al piano.
Il tavolo deve produrre, viste le condizioni di quiete del corpo, una forza uguale e contraria alla forza di attrazione terrestre che chiamiamo reazione vincolare \(\overrightarrow{N}\).

In generale la reazione vincolare non è determinabile a priori, utilizzando una data formula, ma deve essere calcolata caso per caso dall'esame delle condizioni fisiche. 

\end{document}