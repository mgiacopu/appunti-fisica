\section{Cinematica}
$$x(t)$$
$$v(t) = \frac{dx}{dt}$$
$$a(t) = \frac{dv}{dt} = \frac{d^2x}{dt^2}$$
	\subsection{Moto rettilineo}
	\begin{itemize}
		\item Velocit\`a media: $v_m = \frac{\Delta x}{\Delta t}$
		\item Velocit\`a istantanea: $v = \frac{dx}{dt}$
		\item Accelerazione media: $a_m = \frac{dv}{dt}$
		\item Accelerazione istantanea: $a = \frac{dv}{dt}=\frac{d^2x}{dt^2}$
		\item Derivando $x(t)$ si ottiene la velocit\`a istantanea.
		\item Nota velocit\`a legge oraria: $x(t) = x_o + \int_{t_0}^{t}v(t)dt$
		\item Velocit\`a media e istantanea: $v_m = \frac{1}{t-t_0}\int_{t_0}^t v(t)dt$
		\item Velocit\`a nota accelerazione: $v(t) = v_0+\int_{t_0}^t a(t)dt$
		\item Dipendenza di accelerazione da posizione: $adx =vdv$ $\int_{t_0}^t a(x)dx = \frac{1}{2}v^2-\frac{1}{2}v_0^2$
		\item Moto uniformemente accelerato: $v^2 = v_0^2 + 2a(x-x_0)$
	\end{itemize}
	\subsection{Moto rettilineo uniforme}
	\begin{itemize}
		\item $x(t) = x_0 + v(t-t_0)$
	\end{itemize}
	\subsection{Moto rettilineo uniformemente accelerato}
	\begin{itemize}
		\item $x(t) = x_0 + tv_0 + \frac{1}{2}t^2a$
		\item $v(t) = v_0+at$.
	\end{itemize}
	\subsection{Moto verticale}
	\begin{itemize}
		\item $g=-9.81$
	\end{itemize}
	\subsection{Moto armonico semplice}
	\begin{itemize}
		\item $x(t)=A\sin(\omega t+\phi)$
		\item $T = \frac{2\pi}{\omega}$
		\item $\omega = 2\pi v$
		\item $v = \omega A\cos(\omega t+\phi)$
		\item $a = -\omega^2 x$
		\item $x_0 = A\sin\phi$ $v_0 = \omega A\cos \phi$
		\item $\tan\phi = \frac{\omega x_0}{v_0}$
		\item $A^2 = x_0^2 + \frac{v_0^2}{\omega^2}$
		\item $v(x) = \frac{1}{2}v^2-\frac{1}{2}v_0^2$
		\item $v^2 = v_0^2+\omega^2(x_0^2(x_0^2 - x^2))$
		\item $v^2(x) = \omega^2(A^2-x^2)$
	\end{itemize}
	\subsection{Moto nel piano}
	\begin{itemize}
		\item $x = r\cos\theta$ $y = r \sin \theta$
		\item $r = \sqrt{x^2+y^2}$ $\tan\theta=\frac{y}{x}$
		\item $\overrightarrow{r}(t)=x(t)\overrightarrow{u}_x + y(y)\overrightarrow{u}_y$
		\item $v = \frac{d\overrightarrow{r}}{dt}$
		\item $\overrightarrow{v} = v\overrightarrow{u}_T$
		\item $v=v_x\overrightarrow{u}_x+v_y\overrightarrow{u}_y$
		\item $v = \sqrt{v_x^2+v_y^2}$
		\item $\tan\phi = \frac{v_y}{v_x}$
		\item $v = v_r+v_\theta$
		\item $v = \sqrt{\bigl(\frac{d\overrightarrow{r}}{dt}\bigr)^2 + \overrightarrow{r}^2\bigl(\frac{d\theta}{dt}\bigr)^2}$
		\item $\overrightarrow{r}(t)=\overrightarrow{r}(t_o)+\int_{t_0}^tv(t)dt$ scomponendo su componenti.
		\item $\overrightarrow{a} = \frac{d\overrightarrow{v}}{dt}=\frac{d^2\overrightarrow{r}}{dt^2}$
		\item $\overrightarrow{a} = \frac{dv}{dt}\overrightarrow{u}_T + v\frac{d\theta}{dt}\overrightarrow{u}_N$
		\item Componente normale $\overrightarrow{a} = \overrightarrow{a}_T+\overrightarrow{a}_N$
		\item $a = \sqrt{a^2_T+a^2_N} = \sqrt{\bigl(\frac{d\overrightarrow{v}}{dt}\bigr)^2+\frac{v^4}{R^2}}$
		\item Moto curvilineo uniforme $\overrightarrow{a} = a_x\overrightarrow{u}_x+a_y\overrightarrow{u}_y$, $\overrightarrow{a}_x = \frac{d\overrightarrow{v}}{dt}\cos\phi - \frac{\overrightarrow{v}^2}{R}\sin\phi$, $\overrightarrow{a}_y = \frac{d\overrightarrow{v}}{dt}\sin\phi - \frac{\overrightarrow{v}^2}{R}\cos\phi$, $\overrightarrow{a} = \bigl[\frac{d^2\overrightarrow{r}}{dt^2}-r\bigl(\frac{d\theta}{dt}\bigr)^2\bigr]\overrightarrow{u}_r+\bigl[2\frac{1}{r}\frac{d}{dt}\bigl(\overrightarrow{r}^2\frac{d\theta}{dt}\bigr)\bigr]\overrightarrow{u}_\theta$ $\overrightarrow{v}(t) = v(t_0)+\int_{t_0}^t\overrightarrow{a}(t)dt$
	\end{itemize}
	\subsection{Moto circolare}
	\begin{itemize}
		\item $x(t) = R\cos\theta(t)$ $y(t) = R\sin\theta(t)$
		\item $\omega = \frac{v}{R}$
	\end{itemize}
	\subsection{Moto circolare uniforme}
	\begin{itemize}
		\item $s(t) = s_0 + vt$ $\theta(t) = \theta+\omega t$
		\item $a = a_N = \omega^2R$
		\item $T = \frac{2\pi}{\omega}$
		\item $x = R\cos(\omega t+\omega_0)$ $y = R\sin(\omega t+\omega_0)$
	\end{itemize}
	\subsection{Moto circolare non uniforme}
	\begin{itemize}
		\item $a_T=\frac{dv}{dt}$
		\item $a = \frac{a_T}{R}$
		\item $\omega(t)=\omega_0+\int_{t_0}^t a(t)dt$ $\theta(t) = \theta_0 + \int_{t_0}^t\omega(t)dt$
		\item $a = \frac{1}{2}\omega^2-\frac{1}{2}\omega^2_0$
		\item Uniformemente accelerato $\omega = \omega_0+at$ $\theta = \theta_0 + \omega_0+\frac{1}{2}at^2$ $a_N = (\omega_0+at)^2 R$
	\end{itemize}
	\subsection{Moto parabolico dei corpi}
	\begin{itemize}
		\item $a = g = -g\overrightarrow{u}_X$ $r = 0$ $v = v_0$ $t = 0$
		\item $v(t) = v_0-gt\overrightarrow{u}+x$
		\item $v(t) = v_\theta\cos a\overrightarrow{u}_y + (v_\theta \sin a -gt)\overrightarrow{u}_x$
		\item $v_y = v_0\cos\alpha$ $v_x = v_0\sin\alpha -gt$.
		\item $x = v_0\cos at$ $y = v_0\sin\alpha t-\frac{1}{2}gt^2$
		\item $y(t)=\frac{x}{v_0}\cos\alpha$
		\item $y(x) = x\tan\alpha-\frac{g}{2v_0^2\cos^2\alpha}x^2$
		\item $\tan\phi =\tan\alpha=\frac{g}{v_0^2\cos^2\alpha}x$
		\item $x_G = 2x_M$
		\item $x_M = v_0^2\frac{\cos\alpha\sin\alpha}{g}$
		\item $y(x_M)=\frac{v_0^2\sin^2\alpha}{2g}$
		\item $t_G = 2v_0\sin\frac{\alpha}{g}$
	\end{itemize}
