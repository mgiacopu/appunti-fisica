\documentclass[class=book, crop=false, oneside, 12pt]{standalone}
\usepackage{standalone}
\usepackage{amsmath}
\usepackage{../../style}
% \graphicspath{{./assets/images/}}

% arara: pdflatex: { synctex: yes, shell: yes }
% arara: latexmk: { clean: partial }
\begin{document}

\chapter{Conduttori, dielettrici, energia elettrostatica}

\section{Conduttori in equilibrio}

\subsection{Condizione di equilibrio per un conduttore}

I materiali conduttori sono caratterizzati dal fatto che nel loro interno sono verificate particolari condizioni per cui è possibile il moto di alcune delle cariche che li costituiscono.  
La nostra trattazione si concentra sui conduttori solidi, il cui esempio più tipico sono i metalli: in essi per ogni atomo si hanno uno o più elettroni che sono in pratica separati dal resto dell'atomo e liberi di muoversi nel conduttore. 
Con l'applicazione di un opportuno campo \(\overrightarrow{E}\) si può provocare un moto ordinato di elettroni ovvero dar luogo a una corrente elettrica.

Nei fenomeni elettrostatici però le cariche sono fisse e questa condizione richiede che all'interno di un conduttore il campo debba essere nullo, altrimenti ci sarebbe un moto di cariche, contrariamente all'ipotesi. 
Pertanto lo stato di conduttore in equilibrio elettrostatico è definito dalla condizione: 
\begin{equation*}
    \overrightarrow{E} = 0 \text{ all'interno }
\end{equation*}
Si deve intendere che questa è una condizione media macroscopica. 
Nelle immediate vicinanze dei nuclei ci sono campi molto intensi che tengono legati gli elettroni non liberi; inoltre gli elettroni liberi non sono in quiete ma hanno un moto completamente disordinato di agitazione termica. 
Però in nessun istante c'è un moto ordinato in una certa direzione degli elettroni liberi rispetto agli ioni metallici fissi; si usa per questo parlare di gas di elettroni liberi all'interno di un conduttore.

La condizione \(\overrightarrow{E} = 0\) ha le seguenti conseguenze che caratterizzano un conduttore in equilibrio elettrostatico: 
\begin{itemize}
    \item l'eccesso di carica elettrica in un conduttore può stare solo sulla superficie del conduttore;
    \item il potenziale elettrostatico è costante su tutto il conduttore;
    \item il campo elettrostatico in un punto delle vicinanze della superficie del conduttore è perpendicolare alla superficie e ha intensità \(\sigma/ \epsilon_0\), con \(\sigma\) densità di carica superficiale in quel punto.
\end{itemize}
Per la prima proprietà, se il campo elettrostatico è nullo, è nullo il flusso attraverso una qualunque superficie chiusa \(\Sigma^{\prime}\) tracciata all'interno del conduttore e quindi secondo la legge di Gauss all'interno del conduttore non ci sono cariche ( \(q_{int} = 0\)).
Pertanto l'eccesso di carica si distribuisce sulla superficie del conduttore con densità superficiale \(\sigma = dq/ d \Sigma\); se si cedono al conduttore elettroni questi si portano sulla superficie, se si sottraggono, ne risulta carente lo strato superficiale. 

Il potenziale elettrostatico risulta costante in ogni punto del conduttore perché presi due punti qualsiasi
\begin{equation*}
    V(P_2) - V(P_1) = - \int_{P_1}^{P_2} \overrightarrow{E} \cdot d \overrightarrow{s} = 0 \implies V(P_2) = V(P_1) = V_0
\end{equation*}
Il risultato è vero anche se uno dei due punti sta sulla superficie del conduttore, che risulta quindi essere una \emph{superficie equipotenziale}. 

\subsection{Teorema di Coulomb}

Dato che la superficie del conduttore è equipotenziale, il campo elettrostatico \(\overrightarrow{E}\) in un punto esterno molto vicino al conduttore è ortogonale alla superficie del conduttore, indipendentemente dalla forma di questo.

Il valore di \(\overrightarrow{E}\) si ricava applicando la legge di Gauss ad un cilindro retto di basi \(d \Sigma\) e superficie laterale di area trascurabile rispetto a \(d \Sigma\), con una base contenuta all'interno del conduttore, in cui \(E = 0\), e l'altra in prossimità immediata del conduttore all'esterno, dove il campo elettrostatico \(\overrightarrow{E}\) è normale alla superficie. 
Detta \(dq\) la carica contenuta all'interno, sulla superficie del conduttore, si ha:
\begin{equation*}
    \oint \overrightarrow{E} \cdot \overrightarrow{u}_n d \Sigma = E d \Sigma = \frac{1}{\epsilon_0} d q = \frac{1}{\epsilon_0} \sigma d \Sigma
\end{equation*}
e quindi
\begin{equation} \label{teorema_coulomb}
    \overrightarrow{E} = \frac{\sigma}{\epsilon_0}\overrightarrow{u}_n
\end{equation}
detto anche \emph{Teorema di Coulomb}. 
Il verso è uscente se la densità è positiva, entrante se è negativa. 
Si vede che il modulo del campo elettrostatico è maggiore dove \(\sigma\) è maggiore (questo avviene dove il raggio di curvatura è minore).

Un conduttore carico lontano da altri conduttori ha dunque una distribuzione superficiale di carica tale che il campo elettrostatico all'interno sia nullo, qualunque sia la forma del conduttore. 
In particolare se il conduttore è sferico la carica è distribuita uniformemente.
Notiamo inoltre che la carica deve avere lo stesso segno, positivo o negativo, ovunque sulla superficie: un accumulo di elettroni soltanto in una certa zona sarebbe dovuto esclusivamente a un campo elettrico esterno.

\subsection{Induzione elettrostatica}

Avvicinando un conduttore, carico o scarico, ad un altro corpo carico, ovvero introducendolo in un campo elettrico esterno \(\overrightarrow{E}\), il campo elettrostatico all'interno non sarebbe più nullo, ma sarebbe dato da \(\overrightarrow{E}\); senonché questo fatto provoca un movimento di elettroni che si spostano per l'azione del campo elettrico esterno e si accumulano in una zona della superficie, lasciando sul resto della superficie un eccesso di carica positiva: 
tra queste zone si crea un campo dettrostatico indotto \(\overrightarrow{E}_i\) che contrasta il movimento degli elettroni e si raggiunge l'equilibrio quando \(E + E_i = 0\) in tutto l'interno del conduttore.
Abbiamo così una distribuzione di carica elettrica indotta dei due segni sulla superficie del conduttore che si sovrappone all'eventuale carica elettrica preesistente; in totale però la carica elettrica del conduttore rimane la stessa poiché la carica elettrica indotta è la somma algebrica dei due contributi eguali ed opposti.
È il caso dell'induzione elettrostatica.
%TODO decidere se aggiungere esempi di questo caso

Se poniamo a contatto due o più conduttori, ad esempio collegandoli con un filo conduttore, si costituisce un unico corpo conduttore e in equilibrio vale ovunque la condizione \(E = 0\), \(V =\text{ costante}\): i conduttori a contatto hanno lo stesso potenziale. 

\section{Conduttore cavo, schermo elettrostatico}

\subsection{Conduttore cavo}

Consideriamo un conduttore carico che abbia nel suo interno una cavità all'interno della quale non ci siano cariche elettriche. 
Nella massa del conduttore il campo elettrostatico è nullo e pertanto è nullo il flusso attraverso qualsiasi superficie chiusa, in particolare attraverso qualsiasi superficie chiusa \(\Sigma\) che racchiuda la cavità: segue, per la legge di Gauss, che all'interno di \(\Sigma\) non ci sono cariche e quindi sulle pareti della cavità la carica è nulla.

\subsubsection{Dimostrazione}

Se sulle pareti della cavità fossero presenti due distribuzioni di carica di segno opposto, ci sarebbero nella cavità linee di forza, uscenti dalle cariche positive e entranti in quelle negative. 
La circuitazione di \(\overrightarrow{E}\) lungo una linea chiusa, costituita da un tratto \(C_1\) interno alla cavità su cui \(\overrightarrow{E} \neq 0\) e da un tratto \(C_2\) interno al conduttore dove \(\overrightarrow{E} = 0\), darebbe
\begin{equation*}
    \oint \overrightarrow{E} \cdot d \overrightarrow{s} = \int_{C_1} \overrightarrow{E} \cdot d \overrightarrow{s} + \int_{C_2} \overrightarrow{E} \cdot d \overrightarrow{s} = \int_{C_1} \overrightarrow{E} \cdot d \overrightarrow{s} \neq 0
\end{equation*}
in contrasto col fatto che \(\overrightarrow{E}\) è conservativo. 
Pertanto il campo nella cavità deve es-sere nullo se l'integrale di linea esteso a qualsiasi percorso \(C_1\) interno alla cavità deve essere nullo: sulle pareti della cavità non possono esserci cariche elettriche.
Inoltre è chiaro che il potenziale elettrostatico in un qualsiasi punto della cavità è uguale a quello del conduttore: se ci fosse una differenza di potenziale dovrebbe infatti esserci un campo elettrostatico diverso da zero. 

In conclusione:
\begin{itemize}
    \item la carica di un conduttore in equilibrio elettrostatico si distribuisce sempre e soltanto sulla superficie esterna, anche in presenza di una o più cavità all'interno del conduttore; 
    \item il campo elettrostatico è nullo e il potenziale elettrostatico è costante in ogni punto interno alla superficie del conduttore, anche in presenza di cavità.
\end{itemize}

In particolare un conduttore sferico isolato carico di raggio \(R\), che sia pieno o con una cavità sferica concentrica o con cavità di qualsiasi forma, ha sempre campo elettrostatico nullo all'interno e campo elettrostatico in vicinanza della superficie esterna:
\begin{equation*}
    \overrightarrow{E} = \frac{\sigma}{\epsilon_0} \overrightarrow{u}_n = \frac{q}{4 \pi \epsilon_0 R^2}
\end{equation*}
uguale a quella di una carica puntiforme \(q\) posta nel centro \(O\) della superficie sferica. 

\subsection{Induzione completa}

Consideriamo adesso un conduttore \(C_2\) cavo, isolato e privo di carica, e introduciamo un altro conduttore \(C_1\) carico nella cavità, mantenendolo isolato da \(C_2\). 
In condizioni di equilibrio, se \(C_1\) ha sulla sua superficie esterna una carica \(q\), una carica \(- q\) risulta distribuita sulla superficie interna e una carica \(q\) sulla superficie esterna di \(C_2\).

\subsubsection{Dimostrazione}

Tale fatto si spiega subito con la legge di Gauss: attraverso una superficie chiusa \(\Sigma\) interna a \(C_2\) e contenente la cavità il flusso di \(\overrightarrow{E}\) è nullo in quanto è nullo il campo stesso; di conseguenza all'interno di \(\Sigma\) non c'è carica e se \(C_1\) porta la carica \(q\), sulla superficie interna di \(C_2\) deve necessariamente comparire una carica \(-q\). 
Inoltre essendo \(C_2\) neutro, lo spostamento di una carica \(- q\) sulla superficie interna provoca la comparsa di una carica \(+q\) sulla superficie esterna. 

Siamo di fronte a un fenomeno di induzione che in questo caso, essendo la carica \(q\) completamente contenuta all'interno di una cavità chiusa, si chiama induzione completa: \emph{tutte le linee di forza che partono da \(C_1\) terminano su \(C_2\)}.

Dalla superficie esterna di \(C_2\) partono altre linee di forza, il cui andamento in prossimità del conduttore riflette la distribuzione delle cariche sulla superficie stessa. 
Le due zone in cui esiste un campo sono separate da una zona in cui, in equilibrio, non può esistere campo elettrostatico. 

Il campo elettrostatico all'interno della cavità è determinato dal valore di \(q\), dalla posizione di \(C_1\) e dalla forma geometrica delle due superficie affacciate. 
Però, fissato \(q\), all'esterno l'effetto è sempre lo stesso, qualunque siano forma e  posizione. 
Infatti possiamo dire che l'informazione sulla situazione interna potrebbe passàre all'esterno solo attraverso un campo elettrostatico che penetrasse nel conduttore \(C_2\): ma questo non è possibile per la proprietà dei conduttori in equilibrio di avere campo elettrostatico nullo all'interno. 
Al limite si può portare \(C_1\) a contatto con \(C_2\), con il che le cariche \(+q\) e \(-q\) si elidono, ma all'esterno non cambia nulla: questo fatto ci fa anche capire che la distribuzione della carica \(-q\) sulla faccia interna di \(C_2\) è sempre tale che, sommando l'effetto della carica \(q\) di \(C_1\), il campo elettrostatico dovuto alle cariche nella cavità è nullo all'esterno della cavità.

Analogamente, se variamo la carica sulla superficie esterna oppure variamo la sua distribuzione, ad esempio avvicinando al conduttore un altro corpo carico, cambia il campo elettrostatico all'esterno, ma la distribuzione di carica sulla superficie esterna di \(C_2\) è sempre tale da dare campo elettrostatico nullo all'interno di \(C_2\) e quindi non può alterare il campo locale esistente nella cavità. 

\subsection{Schermo elettrostatico}

Pertanto finché lo spazio interno e lo spazio esterno non sono comunicanti: il conduttore cavo costituisce uno schermo elettrostatico perfetto tra spazio interno e spazio esterno


\end{document}
