\documentclass[class=book, crop=false, oneside, 12pt]{standalone}
\usepackage{standalone}
\usepackage{amsmath}
\usepackage{../../style}
% \graphicspath{{./assets/images/}}

% arara: pdflatex: { synctex: yes, shell: yes }
% arara: latexmk: { clean: partial }
\begin{document}

\chapter{Dinamica dei sistemi di punti materiali}

\section{Sistemi di punti materiali, principio di azione e reazione}

Consideriamo un insieme \(n\) di punti materiali che interagiscono tra loro.\newline
La forza \(\overrightarrow{F}_i\) agente sull'\(i\)-esimo punto si può pensare come risultante delle \emph{forze esterne}
agenti sul punto \(\overrightarrow{F}_i^{(E)}\), e delle forze esercitate dagli altri \(n-1\) punti, forze interne al sistema \(\overrightarrow{F}_i^{(I)}\):
\begin{equation}
    \overrightarrow{F}_i = \overrightarrow{F}_i^{(E)} + \overrightarrow{F}_i^{(I)}
\end{equation}
questa distinzione è utile ma comunque non fa si che il moto sia diviso nelle due forze perchè il moto di \(P_i\) dipende da \(\overrightarrow{F}_i\).

La distinzione tra forze interne ed esterne dipende da come viene definito il sistema di punti, che è arbitraria.

Alle forze interne si applica il principio di azione e reazione, indicato anche come terza legge di Newton. 
Se il punto \(i\)-esimo esercita sul punto \(j\)-esimo la forza \(\overrightarrow{F}_{i,j}\) il punto \(j\)-esimo reagisce esercitando sul punto \(i\)-esimo la forza \(\overrightarrow{F}_{j,i}\)
Si osserva che queste forze hanno la stessa direzione, verso opposto, stesso modulo e stessa retta di azione; esse possono essere attrattive o repulsive. 
% TODO aggiungere disegno

In generale la risultante \(\overrightarrow{F}_i^{(I)}\) delle forze interne agenti sull'\(i\)-esimo punto è diversa da zero, però la risultante di tutte le forze interne del sistema è nulla perché, in base al principio di azione e reazione, esse sono a due a due eguali ed opposte:
\begin{equation}
    \overrightarrow{R}^{(I)} = \sum_i \overrightarrow{F}_i^{(I)} = \sum_{i,j} \overrightarrow{F}_{i,j} = 0 
\end{equation}
con \(i  = 1,2,...,n\) , \(j = 1,2,...,n\) , \(i \neq j\).

\section{Centro di massa di un sistema di punti, Teorema del moto del centro di massa}

\subsection{Centro di massa}

Si definisce come centro di massa di un sistema di punti materiali il punto geometrico la cui posizione è individuata, nel sistema di riferimento considerato, dal raggio vettore:
\begin{equation}
    \overrightarrow{r}_{CM} = \frac{\sum_i {m_i \overrightarrow{r}_i}}{\sum_i m_i} = \frac{m_1 \overrightarrow{r}_1 + m_2 \overrightarrow{r}_2 + ... + m_n \overrightarrow{r}_n}{m_1 + m_2 + ... + m_n}
\end{equation}

Si noti che la posizione del centro di massa rispetto agli \(n\) punti materiali non dipende dal sistema di riferimento, mentre le sue coordinate invece variano a seconda del sistema prescelto.

Se gli \(n\) punti sono in movimento, di norma la posizione del centro di massa varia; sulla base della definizione calcoliamo la velocità del centro di massa:
\begin{equation}
    \overrightarrow{v}_{CM} = \frac{d \overrightarrow{r}_CM}{dt}
\end{equation}

Data \(M = \sum_i m_i\) la massa totale del sistema, vediamo che quindi che \(P\) coincide con la quantità di moto \(M \overrightarrow{v}_{CM}\) del centro di massa, considerato come un punto materiale che abbia la posizione \(\overrightarrow{r}_{CM}\), la velocità \(v_{CM}\) e massa pari alla massa totale \(M\) del sistema.

Analogamente possiamo ricavare l'accelerazione del centro di massa, derivando:
\begin{equation}
    \overrightarrow{a}_{CM} = \frac{d \overrightarrow{v}_{CM}}{dt} = \frac{\sum_i {m_i \frac{d \overrightarrow{v}_i}{dt}}}{\sum_i m_i} = \frac{\sum_i {m_i \overrightarrow{a}_i}}{\sum_i m_i} = \frac{\sum_i m_i \overrightarrow{a}_i}{M}
\end{equation}

Se il sistema di riferimento è inerziale
\begin{equation*}
    m_i \overrightarrow{a}_i = \overrightarrow{F}_i = \overrightarrow{F}_i^{(E)} + \overrightarrow{F}_i^{(I)}
\end{equation*}
secondo (5.1) e sostituendo
\begin{equation}
    M \overrightarrow{a}_{C M}=\sum_i m_{i} \overrightarrow{a}_{i}=\sum_{i}\left(\overrightarrow{F}_{i}^{(E)}+\overrightarrow{F}_{i}^{(I)}\right)=\overrightarrow{R}^{(E)}+\overrightarrow{R}^{(I)}=\overrightarrow{R}^{(E)}
\end{equation}
dato che la risultante delle forze è nulla. 

\subsection{Teorema del moto del centro di massa}

La relazione
\begin{equation}
    \overrightarrow{R}^{(E)} = M \overrightarrow{a}_{C M}
\end{equation}
esprime il teorema del moto del centro di massa. 
Il centro di massa si muove come un punto materiale in cui sia concentrata tutta la massa del sistema e a cui sia applicata la risultante delle forze esterne.

Ho inoltre

\begin{equation}
    \overrightarrow{R}^{(E)} = M \overrightarrow{a}_{C M} = M \frac{d \overrightarrow{v}_{C M}}{dt} = \frac{d}{dt} \left( M \overrightarrow{v}_{C M}\right) = \frac{d \overrightarrow{P}}{dt}
\end{equation}

La risultante delle forze esterne è euguale alla derivata rispello al tempo della quantità di moto totale del sistema. 

Il moto del centro di massa è determinato dunque solo dalle forze esterne. 
L'azione delle forze interne non può modificare lo stato di moto del centro di massa; invece il moto di ciascun punto dipende dall'azione delle forze esterne ed interne agenti su di esso.

\end{document}