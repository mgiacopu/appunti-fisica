\documentclass[class=book, crop=false, oneside, 12pt]{standalone}
\usepackage{standalone}
\usepackage{amsmath}
\usepackage{../../style}
% \graphicspath{{./assets/images/}}

% arara: pdflatex: { synctex: yes, shell: yes }
% arara: latexmk: { clean: partial }
\begin{document}

\chapter{Legge di Gauss}

\section{Angolo solido}

Sia \(d \Sigma\) un elemento di superficie, \(\overrightarrow{u}_n\) la sua normale e \(d \Sigma_0\) la sua funzione ortogonale a \(\overrightarrow{u}_r\), il versore del raggio uscente da un punto di osservazione \(0\). 
Si definisce angolo solido infinitesimo, la quantità
\begin{equation}
    d \Omega = \frac{d \Sigma \cos \alpha}{r^2} = \frac{d \Sigma_0}{r^2}
\end{equation}

L'angolo solido è l'estensione a tre dimensioni del concetto di angolo piano infinitesimo:
\begin{equation}
    d \theta = \frac{ds' \cos \alpha}{r} = \frac{ds}{r}
\end{equation}
La superficie \(d \Sigma_0\) è un elemento di calotta sferica e la sua area vale (nel sistema a coordinate polari)
\begin{equation} \label{misura_infinitesimo_angolo_solido}
    d \Omega = \frac{d \Sigma_0}{r^2} = \sin \theta d \theta d \phi
\end{equation}

La (\ref{misura_infinitesimo_angolo_solido}) esprime l'angolo solido sotto cui dal punto \(0\) si vede il contorno \(ABCD\) della superficie \(d \Sigma_0\): 
risulta che \(d \Omega\) non dipende dal raggio \(r\). 
Geometricamente, possiamo dire che l'angolo solido dà una misura della parte di spazio compresa entro un fascio di semirette uscenti da \(0\), così come l'angolo piano dà una misura della parte di piano compresa tra due semirette uscenti da \(0\).

Per una superficie finita l'angolo solido è dato dall'integrale
\begin{equation}
    \Omega \sin \theta d \theta d \phi
\end{equation}
che è un integrale doppio sulle variabili \(\theta\) e \(\phi\).

Se, a \(0\) costante, si fa variare \(\phi\) da zero a \(2 \pi\), abbiamo l'angolo solido sotto cui da \(0\) è vista la corona sferica infinitesima
\begin{equation}
    d \Omega = \sin \theta d \theta \int_0^{2 \pi} d \phi = 2 \pi \sin \theta d \theta
\end{equation}
Integrando da \(\theta_1\) a \(\theta_2\)
\begin{equation*}
    \Omega (\theta_1 , \theta_2) = 2 \pi \left(\cos \theta_1 - \cos \theta_2 \right)
\end{equation*}
In particolare se \(\theta_1 = 0\) e \(\theta_2 = \theta\)
\begin{equation}
    \Omega (\theta) = 2 \pi (1- \cos \theta)
\end{equation}
e infine, se \(\theta_2 = \pi\), abbiamo l'angolo solido sotto cui dal centro è vista tutta la superficie sferica, 
\begin{equation}
    \Omega = 4 \pi
\end{equation}

\section{Flusso del campo elettrostatico, Legge di Gauss}

\subsection{Definizione di flusso attraverso una superficie}

Consideriamo, una superficie \( d \Sigma\) in una regione in cui è definito un campo \(\overrightarrow{E}\) e orientiamola fissando il verso del versore della normale \(\overrightarrow{u}_n\). 
Si definisce flusso del campo \(\overrightarrow{E}\) attraverso la superficie \(d \Sigma\) la quantità scalare 
\begin{equation}
    d \Phi (\overrightarrow{E}) = \overrightarrow{E} \cdot \overrightarrow{u}_n d \Sigma = E \cos \theta d \Sigma = E_n d \Sigma
\end{equation}

Il flusso attraverso una superficie finita \(\Sigma\), si ottiene suddividendo la superficie \(\Sigma\) in elementi infinitesimi di superficie \(d \Sigma_i\), calcolando per ciascuno di essi il flusso infinitesimo \(d \Phi (\overrightarrow{E}_i) = \overrightarrow{E}_i \cdot u_{i,n} d \Sigma_i\), 
e sommando gli infiniti contributi, procedura che porta ad un integrale di superficie:
\begin{equation}
    \Phi (E) = \int_{\Sigma} = \overrightarrow{E} \cdot \overrightarrow{u}_n d \Sigma
\end{equation}
Se la superficie è chiusa, il flusso si scrive, con l'integrale di ciclo chiuso:
\begin{equation} \label{flusso_superficie_chiusa}
    \Phi (E) = \oint_{\Sigma} = \overrightarrow{E} \cdot \overrightarrow{u}_n d \Sigma
\end{equation}

In questo caso è convenzione orientare la normale verso l'esterno. 
I contributi positivi all'integrale (\ref{flusso_superficie_chiusa}) sono quelli per cui \(E \cdot \overrightarrow{u}_n> 0\), dovuti a quelle zone dove anche \(\overrightarrow{E}\) punta verso l'esterno: essi rappresentano un flusso di \(\overrightarrow{E}\) \emph{uscente dalla superficie}. 
I contributi negativi provengono dalle zone in cui \(E \cdot \overrightarrow{u}_n < 0\), in cui cioè \(\overrightarrow{E}\) punta verso l'interno, e rappresentano un flusso di \(\overrightarrow{E}\) \emph{entrante}. 
Pertanto (\ref{flusso_superficie_chiusa}) dà \emph{il flusso netto attraverso la superficie chiusa}; se esso è nullo vuol dire che il flusso entrante eguaglia in modulo il flusso uscente.

\subsection{Legge di Gauss}

La legge di Gauss stabilisce che: 
il flusso del campo elettrostatico \(\overrightarrow{E}\) prodotto da un sistema di cariche attraverso una superficie chiusa è uguale alla somma algebrica delle cariche elettriche contenute all'interno della superficie, divisa per \(\epsilon_0\).
\begin{equation}
    \Phi E = \oint \overrightarrow{E} \cdot \overrightarrow{u}_n d \Sigma = \frac{1}{\epsilon_0} (\Sigma_i q_i)_{int}
\end{equation}
La legge vale indipendentemente da come sono distribuite le cariche all'interno della superficie.
Si noti che il campo \(\overrightarrow{E}\) che compare nell'integrale del flusso è il campo risultante delle tre cariche.
\begin{equation}
    \Phi (E) = \oint \overrightarrow{E} \cdot \overrightarrow{u}_n d \Sigma = \frac{1}{\epsilon_0} \int dq
\end{equation}
essendo l'integrale al terzo membro esteso a tutto il volume \(\tau\) racchiuso da \(\Sigma\).. 
In pratica, poi, l'integrale è esteso ai soli punti in cui c'è carica \(dp \neq 0\). 

Unificando la struttura dell'ultimo membro la legge di Gauss si presenta in maniera generale come:
\begin{equation}
    \Phi (E) = \oint \overrightarrow{E} \cdot \overrightarrow{u}_n d \Sigma = \frac{q}{\epsilon_0}
\end{equation}

\end{document}