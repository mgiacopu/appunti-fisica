\documentclass[class=book, crop=false, oneside, 12pt]{standalone}
\usepackage{standalone}
\usepackage{amsmath}
\usepackage{../../style}
% \graphicspath{{./assets/images/}}

% arara: pdflatex: { synctex: yes, shell: yes }
% arara: latexmk: { clean: partial }
\begin{document}

\chapter{Dinamica dei sistemi di punti materiali}

\section{Sistemi di punti materiali, principio di azione e reazione}

Consideriamo un insieme \(n\) di punti materiali che interagiscono tra loro.\newline
La forza \(\overrightarrow{F}_i\) agente sull'\(i\)-esimo punto si può pensare come risultante delle \emph{forze esterne}
agenti sul punto \(\overrightarrow{F}_i^{(E)}\), e delle forze esercitate dagli altri \(n-1\) punti, forze interne al sistema \(\overrightarrow{F}_i^{(I)}\):
\begin{equation}
    \overrightarrow{F}_i = \overrightarrow{F}_i^{(E)} + \overrightarrow{F}_i^{(I)}
\end{equation}
questa distinzione è utile ma comunque non fa si che il moto sia diviso nelle due forze perchè il moto di \(P_i\) dipende da \(\overrightarrow{F}_i\).

La distinzione tra forze interne ed esterne dipende da come viene definito il sistema di punti, che è arbitraria.

Alle forze interne si applica il principio di azione e reazione, indicato anche come terza legge di Newton. 
Se il punto \(i\)-esimo esercita sul punto \(j\)-esimo la forza \(\overrightarrow{F}_{i,j}\) il punto \(j\)-esimo reagisce esercitando sul punto \(i\)-esimo la forza \(\overrightarrow{F}_{j,i}\)
Si osserva che queste forze hanno la stessa direzione, verso opposto, stesso modulo e stessa retta di azione; esse possono essere attrattive o repulsive. 
% TODO aggiungere disegno

In generale la risultante \(\overrightarrow{F}_i^{(I)}\) delle forze interne agenti sull'\(i\)-esimo punto è diversa da zero, però la risultante di tutte le forze interne del sistema è nulla perché, in base al principio di azione e reazione, esse sono a due a due eguali ed opposte:
\begin{equation}
    \overrightarrow{R}^{(I)} = \sum_i \overrightarrow{F}_i^{(I)} = \sum_{i,j} \overrightarrow{F}_{i,j} = 0 
\end{equation}
con \(i  = 1,2,...,n\) , \(j = 1,2,...,n\) , \(i \neq j\).


\end{document}