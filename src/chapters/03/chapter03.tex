\documentclass[class=book, crop=false, oneside, 12pt]{standalone}
\usepackage{standalone}
\usepackage{amsmath}
\usepackage{../../style}

\begin{document}
\chapter{Moti relativi}

\section{Sistemi di riferimento. Velocità e Accelerazione relative}

Sperimentalmente è provato con estrema accuratezza che le leggi fisiche non dipendono dalla scelta del sistema di riferimento. 
Fissato un sistema di riferimento e stabilita una certa proprietà, questa resta vera anche se cambiano l'origine e l'orientazione degli assi coordinati, ovvero se ci riferiamo ad un altro sistema ottenuto dal primo con una traslazione (spostamento dell'origine, conservando la stessa direzione degli assi) o con una rotazione (stessa origine, cambio della direzione degli assi) o con una operazione combinata.
Non esiste pertanto un punto privilegiato dello spazio e nemmeno un'orientazione privilegiata: lo spazio appare omogeneo e isotropo. 

Abbiamo già rilevato come il concetto stesso di moto sia relativo,  abbia cioè bisogno della precisazione del sistema di riferimento. 

Consideriamo il problema riferendoci alla figura 3.1. 
Il punto \(P\) è in movimento lungo una generica traiettoria. 
Il suo moto viene osservato da una terna cartesiana con centro in \(O\) che, per convenzione, chiamiamo sistema fisso e da una terna cartesiana con centro \(O'\) che, sempre per convenzione, chiamiamo sistema mobile.

Vogliamo ricavare una relazione tra la posizione, la velocità e l'accelerazione del punto \(P\), misurate da un osservatore solidale con il sistema fisso, e le corrispondenti grandezze misurate da un osservatore solidale con il sistema mobile. 

\subsection{Posizione}

La relazione tra le posizioni del punto \(P\), misurate rispetto ai due sistemi di riferimento, è la seguente:
\begin{equation}
    \overrightarrow{r} = \overrightarrow{OO'} + \overrightarrow{r'}
\end{equation}
con 
\begin{equation*}
    \overrightarrow{r} = x \overrightarrow{u_x} + y \overrightarrow{u_y} + z \overrightarrow{u_z} \ , \ \overrightarrow{r'} = x' \overrightarrow{u_{x'}} + y' \overrightarrow{u_{y'}} + z' \overrightarrow{u_{z'}} 
\end{equation*}
Assumiamo, dato che il primo sistema è fisso, che i versori \(\overrightarrow{u_x}, \overrightarrow{u_y}, \overrightarrow{u_z}\) sono indipendenti dal tempo.

\subsection{Velocità}

La velocità del punto \(P\) rispetto al sistema fisso, che chiamiamo velocità assoluta, è data da:
\begin{equation}
    \overrightarrow{v} = \frac{d \overrightarrow{r}}{dt} = \frac{dx}{dt} \overrightarrow{u_x} + frac{dy}{dt} \overrightarrow{u_y} + frac{dz}{dt} \overrightarrow{u_z}
\end{equation}
mentre quella misurata da un osservatore nel sistema mobile, che indichiamo come velocità relativa è
\begin{equation}
    \overrightarrow{v'} = \frac{dx'}{dt} \overrightarrow{u_{x'}} +\frac{dy'}{dt} \overrightarrow{u_{y'}} + \frac{dz'}{dt} \overrightarrow{u_{z'}}
\end{equation}
Infine la velocità dell origine \(O'\) del sistema di riferimento mobile misurata da un osservatore del sistema fisso e data da:
\begin{equation}
    \overrightarrow{v_{O'}}=\frac{d \overrightarrow{OO'}}{d t}=\frac{d x_{O'}}{d t} \overrightarrow{u_{x}}+\frac{d y_{O'}}{d t} \overrightarrow{u_{y}}+\frac{d z_{O'}}{d t} \overrightarrow{u_{z}}
\end{equation}

Derivando rispetto il tempo la (3.1) ottengo

\begin{align*}
    \overrightarrow{v}&=\frac{d \overrightarrow{r}}{dt} = \frac{d \overrightarrow{OO'}}{d t} + \frac{d \overrightarrow{r'}}{dt} = \frac{d x_{O'}}{d t} \overrightarrow{u_{x}}+\frac{d y_{O'}}{d t} \overrightarrow{u_{y}}+\frac{d z_{O'}}{d t} \overrightarrow{u_{z}}\\
    &+\frac{dx'}{dt} \overrightarrow{u_{x'}} + \frac{dy'}{dt} \overrightarrow{u_{y'}} + \frac{dz'}{dt} \overrightarrow{u_{z'}} + x' \frac{d\overrightarrow{u_{x'}}}{dt}  + y' \frac{d\overrightarrow{u_{y'}}}{dt}  + z' \frac{d\overrightarrow{u_{z'}}}{dt}\\
\end{align*}
ovvero
\begin{equation}
    \overrightarrow{v} = \overrightarrow{v_{O'}} + \overrightarrow{v'} + x' \frac{d\overrightarrow{u_{x'}}}{dt}  + y' \frac{d\overrightarrow{u_{y'}}}{dt}  + z' \frac{d\overrightarrow{u_{z'}}}{dt}
\end{equation}

La derivata di un versore \(\overrightarrow{u}\), in quanto vettore con modulo costante, si può scrivere \(\omega \times \overrightarrow{u}\) ; pertanto per le derivate dei tre versori \(\overrightarrow{u_x},\overrightarrow{u_y},\overrightarrow{u_z}\), si hanno le seguenti formule, dette di Poisson:
\begin{equation}
    \frac{d \overrightarrow{u_{x}}}{d t}=\omega \times \overrightarrow{u_{x}} \quad, \quad \frac{d \mathbf{u_{y'}}}{d t}=\omega \times \overrightarrow{u_{y}}, \quad, \quad \frac{d \overrightarrow{u_{z'}}}{d t}=\omega \times \overrightarrow{u_{z}}
\end{equation} 

Posso riscrivere gli ultimi temini dell'equazione della velocità come:
\begin{equation}
    x' (\omega \overrightarrow{u_{x'}}) + y' (\omega \overrightarrow{u_{y'}}) + z' (\omega \overrightarrow{u_{z'}}) = \omega \times (x' \overrightarrow{u_{x'}} + y' \overrightarrow{u_{y'}} + z' \overrightarrow{u_{z'}}) = \omega \times \overrightarrow{r'}
\end{equation}

\subsubsection{Teorema delle velocità relative}
Effettuando l'ultima sostituzione ottengo infine:

\begin{equation}
    \overrightarrow{v} = \overrightarrow{v_{O'}} + \overrightarrow{v'} + \omega \times \overrightarrow{r'}
\end{equation}

La differenza delle due velocità misurate nei sistemi di riferimento è chiamata \emph{velocità di trascinamento}
\begin{equation*}
    \overrightarrow{v_t} = \overrightarrow{v} - \overrightarrow{v'} = \overrightarrow{v_{O'}} + \omega \times \overrightarrow{r'}
\end{equation*}
La velocità di trascinamento è la velocità che il punto mobile \(P\) avrebbe se, nell’istante considerato, fosse solidale con il sistema relativo.

Il moto di trascinamento, legato in pratica al moto del sistema mobile. può essere considerato in ogni istante come la somma di un termine traslatorio con velocità istantanea \(v_{O'}\) e di un termine rotatorio con velocità angolare \(\omega\), variabile in generale sia in modulo che in direzione.

\subsection{Accelerazione}

\end{document}